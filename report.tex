\documentclass[a4paper]{report}
\usepackage{hyperref}
\title{Lab Report for Software Engineering course \newline
 Lab 1: Git \& Dev Cloud}
\author{Wang, Chen \\ (ID: 16307110064) \\ School of Software, Fudan University}
\date{\today}
\bibliographystyle{plain}
\begin{document}
\maketitle

\tableofcontents

\chapter{Background Knowledge of the lab}
\section{Version Control System}
\subsection{The application of VCS}
\subsection{The general usage of VCS}
\subsection{Various tools for VCS}
\section{Git}
\subsection{The feature of Git in comparison to other VCS tools}
\subsection{The application of Git}
\subsection{The general usage of Git}
\subsection{The usage of Git in this lab}
AAAW
\section{Java EE}
\subsection{The needs and application of Java EE}
\subsection{Common frameworks of Java EE}
\section{Spring Boot}
\subsection{The feature of Spring Boot in comparison to other Java EE frameworks}
\subsection{The usage of Spring Boot in this lab}
\section{Classroom platform of Huawei Cloud}
\subsection{The functionalities of Huawei Cloud}
\subsection{The functions of Huawei Cloud used in this lab}
\section{Life cycle in software engineering}
\subsection{The meaning of life cycle in software engineering}
\subsection{The life cycle displayed in this lab}
AAAW
\chapter{Specification of the Lab}
\section{Platform of the operation}
\section{Guideline of the operation steps}
\section{Hand in method and materials}
\chapter{Steps of accomplishing this Lab}
\section{Git operations}
\section{SpringBoot framework construction}
\section{Executions on the SpringBoot framework}
\chapter{Significance of different parts of the lab}
\chapter{Conclusion}
\end{document}